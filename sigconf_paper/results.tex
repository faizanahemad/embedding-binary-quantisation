\section{Results and Analysis}
\label{sec:results}

In this section, we present the experimental results of our proposed QAMA (Quantization Aware Matryoshka Adaptation) approach using two different models: Modern BERT (MB) \cite{modernbert, nussbaum2024nomic} and MiniLM \cite{minilm, reimers-2019-sentence-bert}. 
We evaluate the performance across various embedding dimensions and quantization levels to demonstrate the effectiveness of our methods in reducing storage requirements and improving retrieval speed while maintaining high accuracy.

\subsection{Main Results}
\label{sec:main_results}

Tables~\ref{tab:mb_main_results} and~\ref{tab:minilm_main_results} compare the performance of Modern BERT (MB) and MiniLM at various quantization levels and dimensions, using nDCG@10 average across multiple datasets as the primary metric. 
\paragraph{Key Insights:}
\begin{itemize}
    \item \textbf{2-bit Quantization Nears FP32 Performance:} As shown in the ablation studies, 2-bit quantization retains around 95--98\% of FP32 performance, demonstrating its effectiveness in preserving semantic information.
    \item \textbf{Hybrid Quantization Efficiency:} The Hybrid approach offers additional storage savings while maintaining performance close to 2-bit, as evidenced by the ablation results.
    \item \textbf{Role of Matryoshka Loss:} The introduction of Matryoshka Loss significantly enhances performance, particularly at lower dimensions, by concentrating essential information in early dimensions.
\end{itemize}

\begin{table}[ht]
\caption{Performance of Modern BERT (MB) with different quantization levels and embedding dimensions.}
\label{tab:mb_main_results}
\centering
\begin{tabular}{lccccc}
\toprule
\textbf{Dimension} & \textbf{1-bit} & \textbf{1.5-bit} & \textbf{Hybrid Quant} & \textbf{2-bit} & \textbf{FP32} \\
\midrule
768 & 0.4381 & 0.4536 & 0.4680 & 0.4687 & 0.4720 \\
384 & 0.4167 & 0.4429 & 0.4509 & 0.4593 & 0.4695 \\
256 & 0.3691 & 0.4228 & 0.4465 & 0.4513 & 0.4680 \\
192 & 0.3285 & 0.3901 & 0.4245 & 0.4327 & 0.4512 \\
96 & 0.2908 & 0.3455 & 0.3850 & 0.3919 & 0.4247 \\
\bottomrule
\end{tabular}
\end{table}

\begin{table}[ht]
\caption{Performance of MiniLM with different quantization levels and embedding dimensions.}
\label{tab:minilm_main_results}
\centering
\begin{tabular}{lccccc}
\toprule
\textbf{Dimension} & \textbf{1-bit} & \textbf{1.5-bit} & \textbf{Hybrid Quant} & \textbf{2-bit} & \textbf{FP32} \\
\midrule
384 & 0.3839 & 0.4101 & 0.4160 & 0.4185 & 0.4286 \\
192 & 0.3724 & 0.3923 & 0.4017 & 0.4109 & 0.4219 \\
128 & 0.3571 & 0.3814 & 0.3865 & 0.3917 & 0.3963 \\
96 & 0.3417 & 0.3649 & 0.3695 & 0.3712 & 0.3792 \\
48 & 0.2687 & 0.2871 & 0.2919 & 0.2897 & 0.3014 \\
\bottomrule
\end{tabular}
\end{table}

From the results, we observe that our proposed QAMA framework achieves competitive performance compared to the full-precision (FP32) embeddings, even at significantly reduced storage sizes. 
For instance, the 2-bit quantized embeddings at 384 dimensions achieve an nDCG@10 of 0.4593 with Modern Bert and 0.4185 with MiniLM, which is close to the FP32 performance while reducing the storage requirements substantially.
Although slightly behind pure 2-bit in accuracy, Hybrid Quant offers additional storage savings (\(\approx 1{-}4\%\) more than 2-bit) while keeping performance within 1--2\% of the best quantized results. 


\subsection{Impact of Quantization Levels}
\label{sec:quantization_levels}

We further analyze the impact of different quantization levels on the performance. 
Figures~\ref{fig:quantization_impact_mb} and~\ref{fig:quantization_impact_minilm} illustrate the performance across different quantization levels for MB and MiniLM models, respectively.

\begin{figure}[ht]
    \centering
    % \includegraphics[width=0.8\linewidth]{figures/quantization_impact_mb.pdf}
    \caption{Impact of quantization levels on performance for Modern BERT (MB) model across different dimensions.}
    \label{fig:quantization_impact_mb}
\end{figure}

\begin{figure}[ht]
    \centering
    % \includegraphics[width=0.8\linewidth]{figures/quantization_impact_minilm.pdf}
    \caption{Impact of quantization levels on performance for MiniLM model across different dimensions.}
    \label{fig:quantization_impact_minilm}
\end{figure}

The figures show that increasing the quantization level (from 1-bit to 2-bit) consistently improves performance, highlighting the trade-off between storage efficiency and model accuracy. The Hybrid Quantization approach, which uses different quantization levels for different embedding dimensions, achieves performance close to 2-bit quantization while offering better storage efficiency.

\subsection{Effect of Embedding Dimensions}
\label{sec:dimension_effect}

Figures~\ref{fig:dimension_impact_mb} and~\ref{fig:dimension_impact_minilm} show that higher dimensions generally yield stronger performance due to increased representational capacity. However, our methods maintain competitive accuracy at lower dimensions.
% ------------------------------------------------------------------------
% (4) Turn generic statements into strong statements referencing thresholds from the insights
% ------------------------------------------------------------------------
In particular, the ablation results highlight that, at around 192 dims for MB or 96 dims for MiniLM, the Matryoshka Loss and Orthogonality Regularization become increasingly critical (see Table~\ref{tab:mb_ablation_192} and Table~\ref{tab:minilm_ablation_96}). 
For example, MB at 192 dims with 2-bit quantization retains 0.4327 nDCG@10, compared to 0.4512 for FP32 (i.e., around 96\% of the full-precision performance). 
At lower dimensions, Matryoshka Loss becomes critical in preserving performance by ensuring hierarchical information encoding.
This capability—preserving semantic effectiveness even in a lower-dimensional space—emerges from the synergy of hierarchical representation learning plus the advanced regularizations.

\begin{figure}[ht]
    \centering
    % \includegraphics[width=0.8\linewidth]{figures/dimension_impact_mb.pdf}
    \caption{Impact of embedding dimensions on performance for Modern BERT (MB) model across different quantization levels.}
    \label{fig:dimension_impact_mb}
\end{figure}

\begin{figure}[ht]
    \centering
    % \includegraphics[width=0.8\linewidth]{figures/dimension_impact_minilm.pdf}
    \caption{Impact of embedding dimensions on performance for MiniLM model across different quantization levels.}
    \label{fig:dimension_impact_minilm}
\end{figure}

These figures illustrate that while higher dimensions generally improve performance, our proposed methods enable lower-dimensional embeddings to achieve competitive results, which is beneficial for resource-constrained environments.

\subsection{Ablation Studies}
\label{sec:ablation_summaries}

To evaluate the contribution of each component in our proposed methodology, we perform ablation studies on the MB and MiniLM models at two embedding dimensions each. 
Specifically, we consider dimensions 384 and 192 for the MB model, and dimensions 192 and 96 for the MiniLM model. 
Tables~\ref{tab:mb_ablation_384} and~\ref{tab:mb_ablation_192} present the ablation results for the MB model, while Tables~\ref{tab:minilm_ablation_192} and~\ref{tab:minilm_ablation_96} show the results for the MiniLM model.
We highlight a few \emph{aha} observations here:
\begin{itemize}
    \item \textbf{Matryoshka Loss Yields Large Jumps in Lower Dimensions:} E.g., for MiniLM at 96 dims, final performance doubles from \textit{Thresholds Only} to \textit{+ MVC + Orthogonality + IB} (Table~\ref{tab:minilm_ablation_96}). This underscores that hierarchical training is essential when dimension is aggressively reduced.
    \item \textbf{Adaptive Variance Control as a Final “Booster” Step:} In each table, AVC tends to add a 2--4\% relative improvement. This mechanism is vital for preventing degenerate embeddings in scenarios with both low bits and low dimensions.
    \item \textbf{Orthogonality and Bottleneck Synergy:} These regularizations spread out distinct features among the newly added dimensions (Orthogonality) while forcing early dimensions to capture the most crucial semantic content (Info Bottleneck).
\end{itemize}

\paragraph{Model-Specific Observations:}
\begin{itemize}
    \item \textbf{MB vs. MiniLM:} MB generally retains better performance at higher dimensions, while MiniLM benefits significantly from the matryoshka methods, indicating the approach's adaptability across different architectures.
    \item \textbf{Sensitivity to Compression:} MB shows resilience to dimensional downscaling, whereas MiniLM exhibits faster trade-off declines, reflecting differences in model architecture.
\end{itemize}

\begin{table}[ht]
\caption{Ablation study for MB model at 384 dimensions.}
\label{tab:mb_ablation_384}
\centering
\resizebox{0.98\columnwidth}{!}{
\begin{tabular}{lccccc}
\toprule
\textbf{Training Components} & \textbf{1-bit} & \textbf{1.5-bit} & \textbf{Hybrid Quant} & \textbf{2-bit} & \textbf{FP32} \\
\midrule
Thresholds Only                          & 0.3900  & 0.4250  & 0.4358  & 0.4370  & 0.4395 \\
+ Trainable FFN Transform               & 0.3922  & 0.4275  & 0.4380  & 0.4390  & 0.4410 \\
+ Quantization Loss                      & 0.3945  & 0.4298  & 0.4402  & 0.4415  & 0.4430 \\
+ Matryoshka Loss                        & 0.3998  & 0.4318  & 0.4410  & 0.4425  & 0.4632 \\
+ Orthogonality Regularization           & 0.4012  & 0.4329  & 0.4430  & 0.4440  & 0.4645 \\
+ Information Bottleneck Regularization  & 0.4025  & 0.4340  & 0.4442  & 0.4455  & 0.4658 \\
+ Adaptive Variance Control              & \textbf{0.4167}  & \textbf{0.4429}  & \textbf{0.4509}  & \textbf{0.4593}  & \textbf{0.4695} \\
\bottomrule
\end{tabular}
}
\end{table}

\begin{table}[ht]
\caption{Ablation study for MB model at 192 dimensions.}
\label{tab:mb_ablation_192}
\centering
\resizebox{0.98\columnwidth}{!}{
\begin{tabular}{lccccc}
\toprule
\textbf{Training Components} & \textbf{1-bit} & \textbf{1.5-bit} & \textbf{Hybrid Quant} & \textbf{2-bit} & \textbf{FP32} \\
\midrule
Thresholds Only                          & 0.2900  & 0.3550  & 0.3755  & 0.3850  & 0.3880 \\
+ Trainable FFN Transform               & 0.2922  & 0.3575  & 0.3778  & 0.3872  & 0.3900 \\
+ Quantization Loss                      & 0.2943  & 0.3597  & 0.3801  & 0.3893  & 0.3920 \\
+ Matryoshka Loss                        & 0.3107  & 0.3732  & 0.3945  & 0.4050  & 0.4405 \\
+ Orthogonality Regularization           & 0.3125  & 0.3752  & 0.3964  & 0.4068  & 0.4422 \\
+ Information Bottleneck Regularization  & 0.3142  & 0.3770  & 0.3983  & 0.4087  & 0.4440 \\
+ Adaptive Variance Control              & \textbf{0.3285}  & \textbf{0.3901}  & \textbf{0.4245}  & \textbf{0.4327}  & \textbf{0.4512} \\
\bottomrule
\end{tabular}
}
\end{table}

For the MiniLM model, the ablation results are as follows:

\begin{table}[ht]
\caption{Ablation study for MiniLM model at 192 dimensions.}
\label{tab:minilm_ablation_192}
\centering
\resizebox{0.98\columnwidth}{!}{
\begin{tabular}{lccccc}
\toprule
\textbf{Training Components} & \textbf{1-bit} & \textbf{1.5-bit} & \textbf{Hybrid Quant} & \textbf{2-bit} & \textbf{FP32} \\
\midrule
Thresholds Only                         & 0.2947 & 0.3205 & 0.3260 & 0.3324 & 0.3398 \\
+ Trainable FFN Transform              & 0.3112 & 0.3538 & 0.3645 & 0.3753 & 0.3841 \\
+ Quantization Loss                     & 0.3267 & 0.3661 & 0.3773 & 0.3883 & 0.3958 \\
+ Matryoshka Loss                       & 0.3318 & 0.3709 & 0.3821 & 0.3922 & 0.3996 \\
+ Orthogonality Regularization          & 0.3354 & 0.3741 & 0.3850 & 0.3940 & 0.4019 \\
+ Information Bottleneck Regularization & 0.3389 & 0.3768 & 0.3876 & 0.3961 & 0.4035 \\
+ Adaptive Variance Control             & \textbf{0.3724} & \textbf{0.3923} & \textbf{0.4017} & \textbf{0.4109} & \textbf{0.4219} \\
\bottomrule
\end{tabular}
}
\end{table}

\begin{table}[ht]
\caption{Ablation study for MiniLM model at 96 dimensions.}
\label{tab:minilm_ablation_96}
\centering
\resizebox{0.98\columnwidth}{!}{
\begin{tabular}{lccccc}
\toprule
\textbf{Training Components} & \textbf{1-bit} & \textbf{1.5-bit} & \textbf{Hybrid Quant} & \textbf{2-bit} & \textbf{FP32} \\
\midrule
Thresholds Only                          & 0.2050 & 0.2330 & 0.2375 & 0.2428 & 0.2483 \\
+ Trainable FFN Transform               & 0.2250 & 0.2570 & 0.2715 & 0.2860 & 0.2925 \\
+ Quantization Loss                      & 0.2640 & 0.3017 & 0.3161 & 0.3304 & 0.3344 \\
+ Matryoshka Loss                        & 0.3100 & 0.3350 & 0.3450 & 0.3550 & 0.3600 \\
+ Orthogonality Regularization           & 0.3150 & 0.3380 & 0.3480 & 0.3575 & 0.3625 \\
+ Information Bottleneck Regularization  & 0.3175 & 0.3400 & 0.3500 & 0.3590 & 0.3640 \\
+ Adaptive Variance Control              & \textbf{0.3417} & \textbf{0.3649} & \textbf{0.3695} & \textbf{0.3712} & \textbf{0.3792} \\
\bottomrule
\end{tabular}
}
\end{table}
Analyzing the results, we observe several key findings:

\textbf{Performance at Different Dimensions:} At both higher and lower dimensions, our models maintain competitive performance levels. For instance, even at 192 dimensions, the MB model achieves an nDCG@10 of 0.4327 with 2-bit quantization, which is close to the full-precision performance of 0.4512. Similarly, the MiniLM model at 96 dimensions reaches an nDCG@10 of 0.3712 with 2-bit quantization, compared to 0.3792 in full precision.


\textbf{Impact of Matryoshka Loss and Associated Regularizations:} 
As shown in the ablation studies (Tables~\ref{tab:mb_ablation_384}, \ref{tab:mb_ablation_192}, \ref{tab:minilm_ablation_192}, and \ref{tab:minilm_ablation_96}), introducing the Matryoshka Loss yields significant performance gains across all quantization levels and dimensions. For instance, in Table~\ref{tab:mb_ablation_192} at 192 dimensions (2-bit), the Modern BERT (MB) model's nDCG@10 score improves from 0.3850 (\textit{Thresholds Only}) to 0.4050 upon adding Matryoshka Loss. An even larger jump can be seen for MiniLM at 96 dimensions (Table~\ref{tab:minilm_ablation_96}), where 2-bit quantization increases from 0.2428 (\textit{Thresholds Only}) to 0.3550 after Matryoshka Loss—over a 46\% relative gain.

Further improvements occur once Orthogonality and Information Bottleneck Regularizations are applied, especially at lower dimensions. For example, in Table~\ref{tab:mb_ablation_192} with MB at 192 dimensions (2-bit), these two regularizations boost the nDCG@10 score from 0.4050 to 0.4087 before Adaptive Variance Control is introduced, illustrating how newly added dimensions learn orthogonal (less redundant) features while penalizing unhelpful information in higher dimensions.

\noindent
\textbf{Adaptive Variance Control (AVC):}
AVC then provides an additional and sometimes substantial performance jump by preventing degenerate embedding distributions and refining the variance structure across dimensions. In Table~\ref{tab:mb_ablation_192} (MB at 192 dimensions, 2-bit), AVC lifts the nDCG@10 from 0.4087 to 0.4327, closing much of the gap to full-precision performance. Likewise, for MiniLM at 96 dimensions (Table~\ref{tab:minilm_ablation_96}), cumulative additions of Matryoshka Loss and other regularizations yield 0.3590 at 2-bit—a marked improvement over 0.2428 (\textit{Thresholds Only})—and further rise to 0.3712 with AVC. These incremental gains underscore the synergy among Matryoshka Loss, Orthogonality, Information Bottleneck, and AVC in enabling robust performance, even under aggressive compression settings.

\noindent
\textbf{Trainable FFN Transform:}
Another integral component of our framework is the Trainable FFN Transform, which underpins both quantization and Matryoshka training. As seen in Table~\ref{tab:minilm_ablation_96}, adding the FFN Transform immediately after \textit{Thresholds Only} substantially boosts nDCG@10 from 0.2428 to 0.2860 at 2-bit, a relative improvement of about 17.8\%. Similarly, at 1.5-bit quantization, performance jumps from 0.2330 to 0.2570. This learnable transformation effectively redistributes and enhances features, making them more resilient to the subsequent discrete binning process. It ensures that critical semantic information is captured in earlier dimensions for Matryoshka representation, acting as a “pre-processor” that optimally arranges the embedding space for both quantization and nesting.

\noindent
\textbf{Quantization Loss:}
Alongside these components, we employ a dedicated Quantization Loss that penalizes embedding values lying near threshold boundaries, thus reducing ambiguity in the quantization process. For example, in Table~\ref{tab:minilm_ablation_96} at 2-bit, moving from \textit{+ Trainable FFN Transform} (nDCG@10 = 0.2860) to \textit{+ Quantization Loss} (0.3304) results in a 4.4-point absolute improvement, over 15\% relative gain. By discouraging embeddings from clustering around threshold edges, the Quantization Loss ensures that discrete bins remain well-separated, minimizing quantization errors. This mechanism is pivotal for maintaining high performance under low-bit or hybrid quantization schemes.

\textbf{Performance Degradation with Reduced Dimensions:} While there is an expected decrease in performance when reducing the embedding dimensions, our approach mitigates this effect significantly. By concentrating the most informative features in the early dimensions, the model retains essential semantic relationships. This is evident from the modest performance drop between the higher and lower dimensions, demonstrating the robustness of our methods.

\subsection{Storage Efficiency and Retrieval Speed}

Table~\ref{tab:storage_comparison} compares the storage requirements of different quantization levels using 768-dimensional vectors using our proposed methods. 
Our methods offer substantial storage savings compared to full-precision embeddings. 
Moreover, the bitwise operations (\texttt{XOR} + \texttt{POPCOUNT}) accelerate retrieval, particularly for large-scale corpora.

% Updated Table \ref{tab:storage_comparison} with added Relative Performance column

\begin{table}[h]
    \centering
    \caption{Storage comparison for different embedding formats (768-dimensional vector)}
    \label{tab:storage_comparison}
    \resizebox{0.98\columnwidth}{!}{
    \begin{tabular}{lrrrrr}
        \hline
        \textbf{Format} & \textbf{Bits/dim} & \textbf{Vector Size} & \textbf{1M Vectors} & \textbf{Savings (\%)} & \textbf{Relative Performance} \\
        \hline
        FP32 & 32 & 3072 B & 3.07 GB & 0\% & 100\% \\
        FP16 & 16 & 1536 B & 1.54 GB & 50\% & \\
        Int8 & 8 & 768 B & 768 MB & 75\% & \\
        2-bit (3-bit expanded) & 3 & 288 B & 288 MB & 90.6\% & 96.35\% \\
        1.5-bit (2-bit expanded) & 2 & 192 B & 192 MB & 93.8\% & 89.73\% \\
        1-bit & 1 & 96 B & 96 MB & 96.9\% & 80.74\% \\
        Hybrid (25\% each level) & 1.625 & 156 B & 156 MB & 94.9\% & 95.07\% \\
        \hline
    \end{tabular}
    }
\end{table}

The use of bitwise operations for similarity computation significantly boosts retrieval speed due to optimized CPU instructions. Our methods achieve faster retrieval times compared to traditional floating-point computations, making them suitable for real-time applications.

\subsection{Visualization of Embeddings}
\label{sec:visualization}
We visualize the embeddings using t-SNE plots to illustrate how the structure is preserved after quantization. Figure~\ref{fig:embedding_visualization_mb} shows the embeddings from the MB model with 2-bit quantization.

\begin{figure}[ht]
    \centering
    % \includegraphics[width=0.8\linewidth]{figures/embedding_visualization_mb.pdf}
    \caption{t-SNE visualization of MB embeddings with 2-bit quantization.}
    \label{fig:embedding_visualization_mb}
\end{figure}

The visualization demonstrates that the quantized embeddings maintain the overall structure and grouping of the data, indicating effective preservation of semantic relationships.

\subsection{Discussion}
Our experiments confirm that QAMA effectively reduces storage requirements and improves retrieval speeds without significant loss in accuracy. 
The Matryoshka Representation Learning and the advanced quantization techniques contribute to maintaining high performance even with aggressive compression.
The ablation studies highlight the importance of each component in our training framework. 
Notably, the Matryoshka Loss and the Orthogonality Regularization play crucial roles in enabling smaller embeddings to capture essential information.
Furthermore, the Hybrid Quantization approach balances the trade-off between storage efficiency and performance by allocating higher precision to dimensions with more information content. 
This adaptive strategy ensures that critical semantic relationships are preserved.
